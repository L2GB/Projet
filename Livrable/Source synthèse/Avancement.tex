\documentclass[11pt,a4paper]{report} % Chaque document a une classe (ici report, peut aussi être article ...) définit le type de document. On y met aussi la taille de police en paramètre et le format

\setcounter{tocdepth}{6} % Définit la profondeur du document (sections, lists, captions ...) (Notament pour l'index, savoir jusqu'ou on numérote part, chapter, section, subsection, subsubsection, paragraph ...)
\usepackage[utf8]{inputenc} % Permet de rentrer des caractères directement depuis le clavier
\usepackage[francais]{babel} % Permet a latex de savoir quelle ponctuation utiliser (en fonction de la langue, ex: pas d'espace avant un point et un espace après)
\usepackage[sc]{mathpazo} % Font pour tout ce qui est formule mathématique
\usepackage{amsmath} % Package simplifiant la saisie de fonction mathématique
\usepackage{amsfonts} % Package rajoutant des fonts tenant compte des formules mathématiques
\usepackage{amssymb} % Package permettant l'insertion de symbole mathématique
\usepackage{makeidx} % Package permettant l'automatisation de l'insertion d'un index
\usepackage[toc,page]{appendix} % Package permettant la gestion des annexes
\usepackage{graphicx} % Package permettant l'insertion d'images
\graphicspath{{pictures/}} % Permet de définir un sous dossier ou aller chercher les images à inclure si elles ne sont pas dans le répertoire courrant
\usepackage{longtable} % Package permettant la création de tableaux sur plusieurs pages
\usepackage{perpage} % Package permettant de faire des traitements par page, entre autre la remise à 0 du compteur d'index de footnote
\usepackage{pdflscape}
\usepackage[tight]{shorttoc} % Package permettant de créer une autre table des matière avec une autre profondeur (utile notament si le document est très long et need une table courte en début de document et une plus compléte à la fin par exemple)
\usepackage[]{algorithm2e} % Package permettant l'insertion de bouts d'algo dans un fichier tex (formattage de l'insertion du code)
\usepackage{array} % Permet notamment de modifier le format des colones de tableaux
\usepackage{caption} % Pour faire des listes dans une légende
\usepackage{placeins} % Package permettant de forcer le placement de flotant
\usepackage{framed} % Pour mettre du texte ou autre dans des cadres
\usepackage{float} % Pour créer les flottants (figure, tableau ou autre)
\usepackage{hhline} % Ajoute une autre gestion des tableaux 
\usepackage[table]{xcolor} % Permet de mettre de la couleur dans les tableaux 
\usepackage[format=plain, labelfont=bf, textfont=it]{caption} % Pour modifier l'apparence des légende
\usepackage[left=2cm,right=2cm,top=2cm,bottom=2cm]{geometry} % Définit le format du document
\usepackage{hyperref} %Pour rendre les références clickable and linkable et pour définir leur apparence
\hypersetup{ 
	colorlinks,
	citecolor=black,
	filecolor=black,
	linkcolor=black,
	urlcolor=black
}
\usepackage[toc]{glossaries} % Package permettant la création automatique de glossaire




\usepackage{lscape}




\usepackage{fancyhdr} % Package permettant la création de header et footer plus complexes (Y inclure une image par exemple)
\pagestyle{fancy} % Change le syte des pages qui suivent cette instruction 
\chead{Dossier de conception}
\rhead{}
\lhead{\includegraphics[width=1.5cm]{./Picture/logoEseo}}
\cfoot{\thepage}
\rfoot{Baranger - Bouché - Caillot - Ménard}
\lfoot{}

\newcommand{\sommaire}{\shorttoc{Sommaire}{1}} 

\author{L2GB Inc\\
		ESEO option ASTRE\\
		l2gb.inc@gmail.com
		}

\title{HSMS\\Heating Smart Managing Software  \\Avancement\\L2GB}

 
\makeglossaries % Génère le glossaire
\makeindex % Génère l'index

\MakePerPage{footnote} % Utilisation du package perPage pour ici remettre à 0 à chaque page la numérotation des notes de bas de pages
\usepackage{hyperref} %Pour rendre les références clickable and linkable
\hypersetup{
	colorlinks,
	citecolor=black,
	filecolor=black,
	linkcolor=black,
	urlcolor=black
}
\setcounter{secnumdepth}{5} % Pour que les sections jusqu'au niveau 5 (paragraphe) soient numérotées
\newcommand{\blank}[1]{\hspace*{#1}} % permet de faire des tabulations en tapant \blank
\renewcommand{\figurename}{Illustration} % Permet de changer les légendes des figures, FIGURE <- Illustration

\begin{document} % Début à proprement parler du document
\renewcommand{\figurename}{Illustration} 
\maketitle % Commande indiquant à Latex d'imprimer le titre ici
\sommaire % Commande indiquant à Latex d'imprimer le sommaire iciP

\chapter{Avancement}
		\section{Objet}

Ce document à pour but d'informer les futurs personnes qui travailleront sur le projet de l'état actuel des choses, ce qui est fait et ce qu'il reste à faire en priorité. Il donne aussi des idées pour de futures implémentations.
\newpage

		\section{Etat de l'art}
		
			\subsection{Centrale soft}
			
\begin{description}
	\item[-] Communication avec android fonctionnelle.
	\item[-] La majeure partie des classes sont créées. (Seules les classes des nouveaux objets connectés sont encore à créer)\footnote{Les classes correspondant aux objets de type thermostat ou binary switch sont déjà implémentées}
	\item[-] Création d'un réseau Zwave et ajout d'objets.
	\item[-] Récupération des devices sur le réseau
	\item[-] Récupération des différentes instances des devices connectés
	\item[-] Récupération des différentes commande classe (fonctionnalités) des différentes instances des objets connectés
	\item[-] Pilotage d'une prise et récupération de son état par callback.
	\item[-] Respect d'un planning pour une prise.
\end{description}



			
			
			\subsection{Application Android}
L'application fonctionne actuellement correctement et permet de faire les fonctionnalités de bases dans les cas idéaux:\newline
\begin{description}
	\item[-] Lecture et sauvegarde des données stockées sur la centrale
	\item[-] Communication avec la centrale
	\item[-] Configuration d'objet Prise ou chauffage
	\item[-] Configuration du planning d'une journée
	\item[-] Configuration du planning d'une semaine
	\item[-] Affichage de la consommation des objets enregistrés
	\item[-] Association d'un objet avec un planning semaine
\end{description}

	
\newpage	
		\section{Travail à faire }
		
			\subsection{Centrale soft}
\begin{description}
	\item[-] Agrémenter le protocole de communication
	\item[-] Implémenter toutes les requêtes du protocole
	\item[-] Finir l'implémentation de la classe PowerPlug (prise) (attacher les callbacks pour réagir aux événements voulus)
	\item[-] Implémenter l'instanciation et l'initialisation d'un nouvel objet lorsque celui-ci vient d'être ajouté au réseau
	\item[-] Implémenter les classes relatives à l'utilisation d'autres objets (alarme, porte de garage, tête de chauffage ....)
	\item[-] Utiliser le thermostat
	\item[-] Piloter des radiateurs
	\item[-] Réagir à une modification du scénario en cours 
	\item[-] Améliorer la robustesse
	\item[-] Implémenter la communication vers une base de données
	\item[-] Gestion des objets par pièces
	
\end{description}	

			
			\subsection{Application Android}

\begin{description}
	\item[-] Finir l'application en implémentant toutes les fonctionnalités spécifiées dans le dossier de spécifications. 
	\item[-] Rendre l'application plus fiable, prévoir et agir en fonctions des imprévus.
	\item[-] Tester la vue de consommation des objets une fois celle-ci implémentée sur la centrale.
	\item[-] Implémenter la modification d'un jour.
	\item[-] Implémenter la modification d'une semaine.
	\item[-] Implémenter la gestion des objets par pièce.
	\item[-] Changer la façon de passer d'une activité à une autre.
\end{description}
			
			
\newpage	
		\section{Idées pour de futures implémentations}

\begin{description}
	\item[-] Implémenter l'utilisation d'autres objets connectés.
	\item[-] Réfléchir à l'intégration d'un algorithme prenant en compte la consommation propre de chaque maison, et toutes les données qui peuvent être stockées sur le big data.
	\item[-] Faire des tests de l'algorithme en utilisant la centrale et l'application.
	\item[-] Nouveau design avec plus d'ergonomie.
	\item[-] Mettre en place une interface pour l'agent de maintenance.
\end{description}













		
\end{document}